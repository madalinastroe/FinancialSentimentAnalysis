\chapter{Introducere - Contextul proiectului}\label{ch:context}
\pagestyle{fancy}

\section{Contextul proiectului}\label{sec:context}

\subsection{Context general}\label{subsec:numesub}

Pornim de la acest concept: {\it Finance (în română, finanțe)}. O primă căutare în Google Trends ne oferă o perspectivă
asupra evoluției acestui subiect sau domeniu din ultimii ani. Mai exact, începând cu anul 2004, Google Trends oferă informații referitoare la numărul căutarilor 
ce conțin acest cuvânt cheie.\\

Putem vedea că exista 2 perioade de apogeu: Octombrie 2008 si Martie 2020. În ambele perioade, au existat evenimente care au avut atât consecințe imediate, dar și de lungă durată.
Luând ca exemplu anul 2020, încadrăm la consecințe imediate pierderea locurilor de muncă a persoanelor. Desigur, acest lucru fiind unul temporar și existând soluții, 
oamenii au început să se orienteze spre alte locuri de muncă sau metode de a obține un venit pasiv.
Investițiile în criptomonede au fost cele mai populare. Libertatea oferită de aceste tipuri de investiții, alături de șansele unui profit mare,
într-un context favorabil dezvoltării unor noi hobby-uri, a permis unui grup extins de oameni să încerce să facă lucruri pe care, 
în trecut, doar oamenii care aveau studii în domeniu le puteau înțelege sau realiza.\\

Totuși, libertatea de a alege în ce să investești dintr-o listă vastă de opțiuni putea aduce cu sine un dezavantaj - o decizie neinformată
putea să afecteze negativ persoana în cauză. Partea bună este că, fiind un subiect atât de actual și popular, prezent aproape peste tot în rețelele de socializare, 
reclame, este publicat aproape zilnic un articol legat de aceasta.
Așadar, acest lucru a permis și oamenilor care nu au avut partea de finanțe ca ocupație principală, să își extindă interesul și în această zonă.\\

Tot în această perioadă, pe lângă grupurile de investitori, s-au dezvoltat și aplicații care să ajute persoanele să realizeze tranzacții, să poată observa fluctuația criptomonedelor și acțiunilor,
să compună un istoric cu informațiile necesare pentru ca o persoană să ia decizii informate.

Totuși, un istoric al prețurilor, de multe ori nu explică motivul fluctuației. Valoarea depinde în acest context, de multe ori, de știri și articole.
Un articol perceput ca fiind rău de un investitor, va influența prețul și tranzacțiile viitoare, dar nu e întotdeauna o regulă.

Să luăm următoarele exemple:
\vspace{2mm}
\begin{itemize}
	\item CEO-ul de la renumita companie X s-a retras din activitate, dar urmează să fie inlocuit de Y, care a realizat proiectul Z ce a avut mare succes. 
	\begin{itemize}
		\item Faptul că CEO-ul s-a retras din activitate, ar putea fi perceput ca fiind rău, putând duce la o scădere a prețurilor.
	\end{itemize}
	\begin{itemize}
		\item Faptul că CEO-ul urmează să fie înlocuit de o persoană care a avut succes în domeniu poate fi perceput ca un lucru bun, pentru viitor.
	\end{itemize}
	\begin{itemize}
		\item Analizând punctele de mai sus, șansele de a fi o scădere a prețurilor, urmată de o creștere în viitor sunt destul de mari.
	\end{itemize}
\end{itemize}

\vspace{2mm}

\begin{itemize}
	\item O companie urmează să se extindă, deschizând un sediu într-o altă regiune.
	\begin{itemize}
		\item Știrea despre extindere poate fi percepută ca pozitivă, deci ar putea duce la creșterea prețurilor.
	\end{itemize}
\end{itemize}

\vspace{2mm}

\begin{itemize}
	\item O companie a lansat un produs care a fost sub așteptările clienților, deși, în general compania are produse bune.
	\begin{itemize}
		\item Știrea despre produsul cu defecte poate fi percepută ca fiind negativă.
	\end{itemize}
	\begin{itemize}
		\item Luând în considerare faptul că până acum compania a avut produse ce au satisfăcut clienții, șansele sunt mari ca următorul produs lansat după feedback-ul produsului actual, să fie din nou mulțumitor pentru piață. 
	\end{itemize}
	\begin{itemize}
		\item Analizând punctele de mai sus, există o șansă ca prețurile să scadă acum și să crească în următoarea perioadă, dar există un risc.
	\end{itemize}
\end{itemize}

\subsection{Conturarea domeniului exact al temei}\label{subsec:numesub}
Domeniul investițiilor atât de variabile necesită documentare continuă și atenție ridicată pentru că sunt mulți factori care pot influența, în câteva secunde, o decizie.
Multe dintre aplicațiile create în ultima vreme, s-au concentrat pe a ușura procedurile de tranzacționare pentru utilizatori.\\ 

Ușurința de a face aceste lucruri, la care se adaugă
potențialul câștig, a făcut utilizatorii să urmărească momentul potrivit. Aici putem găsi 2 grupuri: cei care iau deciziile prin precizie matematică și cei care iau deciziile prin studiul pieței.
Persoanele care înțeleg evoluția pe grafuri se ghidează după algoritmi. Celălalt grup înțelege evoluția prin impactul pe care îl au materialele scrise, anume știri, articole, etc.
Dezavantajul major al accesului la atâtea materiale scrise constă în timpul petrecut pentru a-l citi pe fiecare dintre ele, a extrage ideile principale și a-ți forma o decizie financiară informată.\\

Există, desigur, companii mai mari care au diferite instrumente interne pentru a obține o posibilă analiză a sentimentului pieței.
Algoritmii de analiză au o acuratețe ridicată, dar disponibilitatea lor limitată îi face destul de inaccesibil pentru persoanele obișnuite.\\

Așadar, tema proiectului propune o aplicație accesibilă care obține o analiză detaliată a articolelor financiare prin:
\ \\
\begin{itemize}
	\item Alegerea textului de către utilizator
	\item Analiza de sentimente financiare
	\item Extragerea cuvintelor sau frazelor cheie
	\item Sumarizarea textului
\end{itemize}

\ \\
\indent Integrând într-o aplicație funcționalitățile de mai sus, creăm un instrument pentru investitori, fie ei specializați în domeniu sau nu, care oferă o suită de servicii prin care utilizatorii pot să analizeze și să sintetizele articole de știri
ce oferă informații financiare. 
Mai mult decât atât, utilizatorii își pot crea un cont unde își pot salva analizele anterioare pentru a o analiză retrospectivă, dar și pentru a observa topicurile populare.

\section{Structura lucrării}

{\noindent \bf Capitolul 1 - Introducere} - Acest capitol prezintă contextul general și contextul exact al aplicației.\\

{\noindent \bf Capitolul 2 - Obiectivele proiectului} - Acest capitol cuprinde obiectivul principal al proiectului și obiectivele secundare care au dus la realizarea obiectivului proincipal.\\

{\noindent \bf Capitolul 3 - Studiu bibliografic}\\

{\noindent \bf Capitolul 4 - Analiză și Fundamentare teoretică} - În acest capitol sunt descrise principiile funcționale ale aplicației, algoritmii și protocoalele utilizate, explicații ale soluției alese, structura logică și funcțională a aplicației. \\

{\noindent \bf Capitolul 5 - Proiectare de detaliu și Implementare} - În acest capitol sunt prezentate arhitectura, diagramele importante și descrierile componentelor la nivel de modul.\\

{\noindent \bf Capitolul 6 - Testare, Validare și Evaluare} - În acest capitol sunt prezentate metodele de testare și validare ale aplicației.\\

{\noindent \bf Capitolul 7 - Manual de Instalare și Utilizare} - În acest capitol sunt descrișii pașii de instalare ai proiectului și modul de utilizare a aplicației.\\

{\noindent \bf Capitolul 8 - Concluzii}- În acest capitol sunt prezentate rezultatele obținute și descrise posibilele dezvoltări ulterioare.\\

\newpage