\chapter{Obiectivele Proiectului}\label{ch:obiective}
\pagestyle{fancy}

\section{Obiectiul principal}

Obiectivul principal al acestui proiect este de a crea un instrument de analiză amănunțită a unui text financiar. În acest domeniu, cu atenția orientată mai ales spre investiții,
oamenii urmăresc continuu un moment potrivit și sunt predispuși în a lua decizii rapide care îi pot sau nu afecta în viitor.
Deciziile pot fi influențate de mai mulți factori: informațiile transmise prin știri sau articole, cunoștințele precare, etc.
Cu ajutorul acestei aplicații, utilizatorii pot înțelege mai rapid ce este transmis într-un articol voluminos pentru a putea să își formeze o părere informată înainte de a lua o decizie.
Dat fiind faptul că datele de intrare care vor fi folosite pentru a fi analizate în această aplicație sunt majoritatea știri, în aplicație vor fi integrate module care se vor ocupa de înțelegerea și procesarea limbajului natural.



\section{Obiective secundare}
Unul dintre cele mai importante obiective secundare este modul în care este prezentată aplicația utilizatorului final.
Statisticile arată că utilizatorii își doresc, în proporție de 49\%, un design minimalist, intuitiv, cu informațiile principale în secțiuni cât mai ușor accesibile.
Utilizatorul ar trebui să aibă într-un singur loc toate informațiile rezultate în urma analizei unui text financiar, pentru a putea face ușor conexiuni sau comparații.
Pe lângă acestea, utilizatorul poate vedea evoluția pe zile a aparițiilor comune cuvintelor cheie identificate în textul său.\\

Al doilea obiectiv secundar este reprezentat de feedback-ul pe care îl oferă aplicația utilizatorului, prin indicațiile pentru ce trebuie introdus în anumite câmpuri,
notificările primite ca rezultat al anumitor requesturi ale utilizatorului, email de confirmare după crearea unui cont nou, etc.\\
Pentru corectitudinea datelor, dar și pentru protecția lor, utilizatorul primește feedback vizual atunci când datele introduse nu au formatul corect sau numărul de caractere nu este în limita acceptată.\\
De asemenea, utilizatorul este notificat, în cazul introducerii sau modificării datelor personale precum numele de utilizator sau email dacă sunt deja utilizate. \\

Al treilea obiectiv secundar este securitatea aplicației, care va fi asigurată în următoarele moduri: 
\begin{itemize}
    \setlength\itemsep{0.5em}
    \item La crearea unui cont nou, parola salvată în baza de date va fi trecută printr-o funcție de hashing SHA512, generând un output unic de 512 biți.
    \item După crearea cu succes a contului, la logare, utilizatorul va avea un token de acces pentru a putea accesa anumite resurse
    \item Accesul permis doar în anumite pagini ale aplicației, folosind rutele protejate
\end{itemize}
\ \\

Al patrulea obiectiv secundar este reprezentat de abilitatea utilizatorului de modificare a anumitor date din cont, fie ele personale, precum datele contului, strict legate de analizele pe text, adică ce texte analizate 
dorește să pastreze în istoricul personal.
Pentru datele persoanale, singurele restricții sunt impuse pentru datele care identifică un utilizator: username-ul și email-ul; acestea trebuie să fie unice atunci când există o încercare de actualizare a datelor, altfel 
utilizatorul va fi notificat cu un mesaj specific erorii care a apărut în procesul de salvare a datelor.
Pentru partea de salvare în istoricul personal al analizelor efectuate, utilizator poate să aleagă dacă salvează o analiză în istoric, altfel rezultatele sunt decartate.
\\

Al cincilea obiectiv secundar este realizat prin integrarea modelelor de procesare de limbaj natural pentru a fi folosite cu următoarele scopuri:
\begin{itemize}
    \setlength\itemsep{0.5em}
    \item Analiza textului prin extragerea sentimentelor financiare
    \begin{itemize}
        \setlength\itemsep{0.5em}
        \item După procesarea textului financiar, rezultatul analizei constă într-un scor procentual obținut pentru sentimentele negativ, neutru și pozitiv
    \end{itemize}
    \item Sumarizarea textului
    \begin{itemize}
        \setlength\itemsep{0.5em}
        \item Pentru că nu întotdeauna un sentiment pozitiv înseamnă același lucru și în realitate, pentru mai multă claritate utilizatorul va avea extrase ideile principale din textul analizat
    \end{itemize}    
    \item Extragerea de cuvinte cheie
    \begin{itemize}
        \setlength\itemsep{0.5em}
        \item Rezultatul acestei analize a textului constă în cuvintele cheie identificate în text ce vor fi folosite ulterior pentru generarea graficelor de popularitate, la cerința utilizatorului și afișarea definițiilor
    \end{itemize} 
\end{itemize}
\ \\

Alt șaselea obiectiv secundar este de a oferi utilizatorului posibilitatea de a vedea un trending intern, constituit din top 10 cele mai populare cuvinte cheie sau fraze extrase din textele analizate într-o anumită perioadă de timp
selectată de el, dar și evoluția anumitor cuvinte cheie introduse de utilizator. Toate aceste informații vor fi afișate într-un mod cât mai ușor de înțeles de utilizator, conținând informațiile relevante.
Pentru cuvintele cheie extrase, se vor afișa definițiile acestora.\\

Pe lângă toate acestea, aplicația trebuie să fie extensibilă, pentru a putea adăuga funcționalități ulterioare. Proiectarea acesteia trebuie să fie în așa fel încât modulele sau componentele actuale să nu fie afectate de adăugarea de funcționalitate nouă.